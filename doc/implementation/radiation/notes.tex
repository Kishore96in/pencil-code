%$Id: notes.tex,v 1.101 2022/03/06 06:17:42 brandenb Exp $
%\documentclass{article}
\documentclass[twocolumn]{article}
\setlength{\textwidth}{160mm}
\setlength{\oddsidemargin}{-0mm}
\setlength{\textheight}{220mm}
\setlength{\topmargin}{-8mm}
\usepackage{graphicx,natbib,bm,url,color}
\graphicspath{{./fig/}{./png/}}
\thispagestyle{empty}
\input apj
\input gafd_journ
\def\red{\textcolor{red}}
\def\blue{\textcolor{blue}}
\title{}
\author{}
\date{\today,~ $ $Revision: 1.101 $ $}
\begin{document}
%\maketitle

\section{Density unit}

In convection, we used to measure $\rho$ in units of $\rho_0=\bra{\rho}$
and $T$ in units of $[u]^2/\cp$, but in the presence of radiation,
we also have $\sigmaSB$, and thus
\begin{equation}
\sigmaSB \left([u]^2/\cp\right)^4=[\rho][u]^3,
\end{equation}
i.e.,
\begin{equation}
[\rho]=\sigmaSB [u]^5/\cp^4.
\end{equation}
Choosing $[u]=1\km\s^{-1}$, and with
$\cp=3.46\times10^8$ (cgs) and
$\sigmaSB=5.67\times10^{-5}$ (cgs), we have
$[rho]=3.9\times10^{-14}\g\cm^{-3}$.

\section{$\sigmaSB$ as a free parameter}

Fixing instead $[\rho]$, we have,
\begin{equation}
\sigmaSB^{\rm art}=\cp^4\,[\rho]/[u]^5.
\end{equation}
Using now $[\rho]=4\times10^{-4}\g\cm^{-3}$ (BB14), we find
\begin{equation}
\sigmaSB^{\rm art}=5.76\times10^5 \,\mbox{(cgs)}.
\end{equation}
which is $1.02\times10^{10}$ times larger than the actual value.

\section{KH time scale}

\begin{equation}
\tau_{\rm KH}={E\over L}
={[\rho] \cp T R^3\over\sigmaSB^{\rm art} T^4 R^2}
={[\rho] \cp R\over\sigmaSB^{\rm art} T^3}.
\end{equation}
If we replace $T$ by $gR/\cp$ we have
\begin{equation}
\tau_{\rm KH}
={[\rho] \cp R\over\sigmaSB^{\rm art} (gR/\cp)^3}
={[\rho] \cp^4 \over\sigmaSB^{\rm art} g^3 R^2}.
\end{equation}
Replace $R=[u]^2/g$, then
\begin{equation}
\tau_{\rm KH}
={[\rho] \cp^4 \over\sigmaSB^{\rm art} g [u]^4}.
\end{equation}
Using $g=274\times100\cm\s^{-2}$, we have
$\tau_{\rm KH}=1200\yr$.
Decreasing $\tau_{\rm KH}$ can be achieved by making
larger than it is in reality, i.e., the same trend as
found above.

%\begin{figure}[h!]\begin{center}
%\includegraphics[width=\columnwidth]{X}
%\end{center}\caption[]{
%\url{X}
%}\label{X}\end{figure}

%\begin{table}[htb]\caption{
%}\vspace{12pt}\centerline{\begin{tabular}{lccccccc}
%$R$ & $T$ & $\kf$ & $\omf$ & $\EEGW$ & $\hrms$ & $k_{\rm peak}$ \\
%\hline
%\label{Ttimescale}\end{tabular}}\end{table}

%r e f
\begin{thebibliography}{}

\bibitem[Barekat \& Brandenburg(2014)]{BB14}
Barekat, A., \& Brandenburg, A.\yanaN{2014}{571}{A68}
{Near-polytropic stellar simulations with a radiative surface}
(BB14)

\end{thebibliography}

%\vfill\bigskip\noindent\tiny\begin{verbatim}
%$Header: /var/cvs/brandenb/tex/etc/notes.tex,v 1.101 2022/03/06 06:17:42 brandenb Exp $
%\end{verbatim}

\end{document}
